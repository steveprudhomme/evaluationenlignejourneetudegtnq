% Auteur : Steve Prud’Homme
% En ce qui concerne les citations insérées selon le principe de l'utilisation équitable, veuillez les contacter ou respecter les droits d’utilisation précisés dans les documents d’origine avant de les réutiliser.
%Si vous estimez que certains éléments de ce rapport ne respectent pas intégralement les droits de vos
%publications, veuillez nous en aviser afin que les modifications nécessaires puissent être apportées au : sprudhomme@cslaval.qc.ca
% Cette oeuvre, création, site ou texte est sous licence Creative Commons Attribution - Pas d’Utilisation Commerciale - Partage dans les Mêmes Conditions 4.0 International. Pour accéder à une copie de cette licence, merci de vous rendre à l'adresse suivante 
% http://creativecommons.org/licenses/by-nc-sa/4.0/ ou envoyez un courrier à 
% Creative Commons, 444 Castro Street, Suite 900, Mountain View, California, 94041, USA.
\documentclass{beamer}
\usepackage{etex}
\reserveinserts{28}
\usepackage{color}
\usepackage{beamerthemesplit} % new 
\usepackage[french]{babel}
\usepackage[utf8]{inputenc}
\usepackage{tikz}
\usepackage[fixlanguage]{babelbib}
\selectbiblanguage{french}
% Natlib pour la bibliographie
\usepackage{natbib}
\usepackage{url}
\usetikzlibrary{mindmap,shadows,shapes,backgrounds}
\usepackage[T1]{fontenc}
\setbeamertemplate{bibliography item}[text]
\usepackage{multicol}
%\usepackage{pst-barcode}
%\usepackage{auto-pst-pdf} % uncomment this if used with pdflatex


\definecolor{MightySlate}{RGB}{85,98,112}
\definecolor{Pacifica}{RGB}{78,205,196}
\definecolor{AppleChic}{RGB}{199,244,100}
\definecolor{CheeryPink}{RGB}{255,107,107}

\setbeamercolor{titlelike}{parent=structure}
\setbeamerfont*{title}{size=\huge}
\setbeamercolor{title}{bg=MightySlate, fg=white}
\setbeamercolor{author}{bg=Pacifica, fg=white}
\setbeamercolor{institute}{bg=AppleChic, fg=black}
\setbeamercolor{date}{bg=CheeryPink, fg=white}

\definecolor{DTUred}{RGB}{178,20,20}
\setbeamercolor*{palette primary}{use=structure,fg=white,bg=MightySlate}
\usepackage{helvet}
%\usepackage{draftwatermark}
%\SetWatermarkLightness{0.5}
%\SetWatermarkAngle{25}
%\SetWatermarkScale{0.5}
%\SetWatermarkFontSize{2cm}
%\SetWatermarkText{Document de travail}

\begin{document}
	\title{L'évaluation en ligne} 
	\author{Steve Prud'Homme} 
	\institute{GTN-Québec - Commission scolaire de Laval} 
	\date{\today} 

	
	\usebackgroundtemplate{%
  \includegraphics[width=\paperwidth,height=\paperheight]{sommaire.png}} 
	
	\frame{\titlepage} 
	\usebackgroundtemplate{ } 
	
	\par L’intention de ce document est de respecter pleinement les droits des créateurs des ressources
utilisées.
	\par En ce qui concerne les citations insérées selon le principe de l'utilisation équitable ou avec la permission de l'auteur, veuillez les contacter ou respecter les droits d’utilisation précisés dans les documents d’origine avant de les réutiliser.
	\par Si vous estimez que certains éléments de ce rapport ne respectent pas intégralement les droits de vos
publications, veuillez nous en aviser afin que les modifications nécessaires puissent être apportées au : \url{mailto:sprudhomme@cslaval.qc.ca}.
	\par Cette \oe uvre, création, site ou texte est sous licence Creative Commons Attribution\,-\,Pas d’Utilisation Commerciale\,-\,Partage dans les Mêmes Conditions 4.0 International.
	\section{Sommaire} 
		\begin{frame}
			Cette présentation vise à :
			\frametitle{Sommaire}
			\begin {itemize}
				\item \textbf{Familiariser} l'auditoire avec l'évaluation en ligne en présentant un bref aperçu de la \textbf{littérature} québécoise, étatsunienne et britanique sur le sujet.
				\item Effecturer un court \textbf{état des lieux} de l'évaluation en ligne et de ses pratiques en formation professionnelle.
				\item Démontrer qu'il est pertinent d'adopter des \textbf{pratiques harmonisées} en ce qui concerne l'évaluation en ligne.
				\item Promouvoir l'idée qu'il est essentiel que le MEES \textbf{établisse des lignes directrices} rapidement en ce qui concerne l'évaluation en ligne.

			\end{itemize}
		\end{frame}
	\frame[allowframebreaks]{\frametitle{Ordre du jour}\tableofcontents}


	\section{Contexte} 
		\begin{frame}[allowframebreaks]
			\frametitle{Le contexte de la formation à distance et en ligne}
			Le \citet{OLC2015a}, conclus dans une grande étude étatsunienne qu'en 2015\footnote{Infographie : \url {http://www.onlinelearningsurvey.com/reports/2015SurveyInfo.pdf} } :
			\begin {itemize}
				\item \textbf{Hausse d'inscription} en formation à distance de 3,9\,\% par rapport à l'année 2014.
				\item Plus du {quart des étudiants (28\,\%) prend des cours en formation à distance} (un  total de 5\,828\,826 étudiants, une hausse annuelle de 217\,275)  pour une population de 321\,773\,631 (soit 1,8 \,\%).
				\item 71,4\,\% des directeurs universitaires considèrent que les \textbf{résultats de l'apprentissage dans l'éducation en ligne sont égaux ou supérieurs} à ceux de l'enseignement en face à face.
				\item 29,1\,\% des dirigeants universitaires rapportent que leur faculté reconnaît la \textbf{«\,valeur et la légitimité de l'éducation en ligne\,»}. 
				\item Parmi les écoles ayant les inscriptions les géographiquement éloignées, 60,1\,\% rapportent la reconnaissance de la formation en ligne par la faculté. 
				\item Pourtant 11,6\,\% des écoles avec des inscriptions locales font de la formation en ligne. 
			\end{itemize}
			\framebreak
			{Selon la \citet{sofad2015a} en 2014\,:} 
			
			\begin {itemize}
				\item Le total des inscriptions à des cours à distance (formations générales et professionnelles) \textbf{a augmenté de 3,6\,\%} et \textbf{atteint un sommet de 56 608 inscriptions}.
				\item Le nombre d’élèves concernés par ces cours a augmenté de 3\,\% pour atteindre un sommet à 29\,386 pour une population, selon \citet{is2015a}, de 8\,263\,600 (soit 0,36\,\%).
				\item \textbf{La croissance est particulièrement forte en formation professionnelle} à distance avec des augmentations de 8\,\% du nombre d’inscriptions et de 10\,\% pour le nombre d’élèves, pour atteindre de nouveaux sommets dans un cas comme dans l’autre (10\,209 inscriptions et 3\,374 élèves).
				
			\end{itemize}
		\end{frame}
		
		\subsection{La question de l'évaluation en ligne} 
		\begin{frame}
			\frametitle{La question l'évaluation en ligne}
			\begin {itemize}
				\item \textbf{Très peu discutée}. «\,Si le terme fait maintenant partie du vocabulaire courant de la formation, le concept d'évaluation électronique ou en ligne est encore relativement peu répandu\,» \citep{audet2011a} qui cite \citet{buzzetto2006a}. 
				\item Il existe des désaccords à ce sujet.
				
			\end{itemize}
		\end{frame}
		
	\subsection{L'aspect séducteur} 
		\begin{frame}
			\frametitle{L'aspect séducteur de l'évaluation en ligne\citep{Lamontagne2013}}
			\begin {itemize}
				\item Gestion facilitée
				\item Économie
				\item Standardisation
				\item Personnalisation
				\item Contrôle
				\item Rapidité
				\item Flexibilité
				\item Qualité
			\end{itemize}
		\end{frame}
	
	\subsection{Résultats} 
		\begin{frame}
			\frametitle{Des résultats ? \citep{Lamontagne2013} tiré de \citet {Stansbury2013C}}
			
			\begin {itemize}
				\item L'élève-étudiant réussi mieux aux tests d'état finaux (\textbf{+20\,\% sur 6 ans}).
				\item \textbf{Un taux de poursuite aux études supérieures de + 40\,\%} comparer en traditionnel.
				\item \textbf{Une augmentation de 16\,\% dans la réussite globale} comparer en traditionnel.
			\end{itemize}
			
			\par Ces améliorations sont dues à la \textbf{systématisation des suivis} et de \textbf{l'attention accordée à chacun}.
			\par \alert{Un changement induit par l'utilisation de la formation en ligne}. 
			
		\end{frame}
	
	\section{Facteurs} 
		
		\subsection{Les facteurs liés à la formation ou à l'aisance TIC} 
			\begin{frame}[allowframebreaks]
			 	\frametitle{Les facteurs liés à la formation ou à l'aisance TIC \citep{NorthCarolina2013} et \citep{Lamontagne2013}}
				\begin {itemize}
					\item Préparer le personnel enseignant et administratif\,:
						\begin {itemize}
							\item inclut la formation des responsables TI,
							\item inclut la formation des administrateurs et gestionnaires de tests à tous les niveaux (local, régional, provincial)\,;
						\end{itemize}	
					\item Préparer les élèves ou étudiants\,:
						\begin {itemize}
							\item former les étudiants pour l'utilisation des outils\,;
						\end{itemize}
					\item S'approprier les connaissances de base et mettre en place un support national \citep{Stansbury2013A}\,;
					\item Permettre une implantation nationale de l'infrastructure et de l'évaluation en ligne ainsi qu'une approche de l'apprentissage personnalisé \citep{Stansbury2013A}.
	
				\end{itemize}
				
			\end{frame}
			
		\subsection{Les facteurs pédagogiques} 
			\begin{frame}[allowframebreaks]
			 	\frametitle{Les facteurs pédagogiques \citet{audet2011a}}
				\begin {itemize}
					\item Les objectifs d'apprentissage, l'évaluation en ligne\,:
						\begin {itemize}
							\item \textbf{facilite l'identification} de ces objectifs, leur \textbf{mise en relation} et leur \textbf{partage} entre concepteurs,
							\item \textbf{soutient leur diffusion} aux élèves\,/\,étudiants et \textbf{clarifient ainsi les attentes} de l'évaluation\,;
						\end{itemize}
					\item Les compétences évaluées, l'évaluation en ligne\,:
						\begin {itemize}
							\item rend \textbf{beaucoup plus simple} la réalisation d'activités d'évaluation visant la démonstration de compétences diversifiées, incluant par exemple la créativité ou la collaboration,
							\item \textbf{vise des compétences de plus haut niveau} que la simple mémorisation\,;
						\end{itemize}
					\item La finalité des évaluations, l'évaluation en ligne\,:
						\begin {itemize}
							\item contribue de façon importante à une {évaluation plus formative et plus formatrice},
							\item \textbf{facilite également l'administration des tests diagnostiques} et \textbf{l'utilisation à plusieurs fins d'un même ensemble de contenus} d'évaluation\,;					
						\end{itemize}
					\framebreak
					\item La rétroaction, l'évaluation en ligne\,: 
						\begin {itemize}
							\item \textbf{facilite et accélère la transmission de la rétroaction},
							\item cette rapidité de rétroaction est déterminante dans le développement de l'évaluation en ligne, car elle permet \citep{whitelock2006a} tiré de \citet{audet2011a}  :
								\begin {itemize}
									\item l'automatisation,
									\item la remise immédiate de la rétroaction à l'apprenant,
									\item la réutilisation\,;
								\end{itemize}
							\item elle permet aussi de \textbf{rendre l'évaluation plus captivante} parce qu'elle permet d'y insérer des éléments \textbf{multimédias} ou des \textbf{hyperliens}\,;
						\end{itemize}
					\framebreak 
					\item Les activités réalisées\,:
						\begin {itemize}
							\item La réalisation d'activités d'évaluation\,:
								\begin {itemize}
									\item plus variées,
									\item plus authentiques,
									\item plus collaboratives\,;
								\end{itemize}
							\item presque que toutes les formes de \textbf{démonstrations de compétences} (exemple : les wikis),
							\item il est facile de \textbf{combiner des évaluations alternatives avec des évaluations} d'aide à l'apprentissage (formatives) ou aux fins de la sanction (sommatives) automatisés (diversité versus tâche de l'enseignant)\,;										
						\end{itemize}
					\framebreak 
					\item Fréquence de la mesure\,:
						\begin {itemize}
							\item Le Web vient alors \textbf{supporter la compilation et l'analyse des divers résultats obtenus} \citep{audet2011a} \,;
						\end{itemize}
					\framebreak 
					\item Les évaluateurs\,:
						\begin {itemize}
							\item les outils d'évaluation en ligne \textbf{facilitent particulièrement les formes alternatives d'évaluation} en permettant\,:
								\begin {itemize}
									\item l'anonymat,
									\item la pondération complexe,
									\item la compilation des résultats,
									\item la discussion,
									\item la négociation,
									\item la conservation des traces\,;
								\end{itemize}
						\end{itemize}
					\framebreak
					\item Les critères d'évaluation\,:
						\begin {itemize}		
							\item l'utilisation de \textbf{critères détaillés facilite par la compilation efficace} de résultats,
							\item l'\textbf{élimination des contraintes liées à la diffusion des documents} (diffusion des critères préalables et des résultats liés)\,;
						\end{itemize}
					\item La portée \citep{jisc2007a}, tiré de \citet{audet2011a}\,:
						\begin {itemize}		
							\item  l'évaluation en ligne est souvent restreinte aux évaluations à portée faible\footnote{\tiny{Habituellement formative et ses résultats demeurent locaux}} et moyenne\footnote{\tiny{Peuvent avoir des résultats locaux ou nationaux, mais ceux-ci ne sanctionnent pas l'apprenant}},
							\item de plus en plus d'organisations envisagent l'évaluation en ligne dans des contextes de \textbf{certification professionnelle}, où l'impact d'un succès ou d'un échec est très important\,;
						\end{itemize}
					\item  nous allons vers un alignement \citep{audet2011a}\,:
						\begin {itemize}		
							\item L'évaluation est plus efficace quand elle reflète une \textbf{compréhension de l'apprentissage comme multidimensionnel}, intégrée et démontrée par une performance dans le temps \citep{audet2011a}, qui se traduit selon  en quatre composante \citep{angelo1996a}\,:
								\begin {itemize}
									\item la multiplicité des méthodes,
									\item les dimensions de l'apprentissage,
									\item les évaluateurs,
									\item les moments d'évaluation\,;
								\end{itemize}
								 \textbf{L'évaluation en ligne facilite la mise en place des ces nombreux paramètres et le suivi qu'ils nécessitent}, en fonction de l'intention pédagogique du concepteur \citep{audet2011a}.
						\end{itemize}
				\end{itemize}
			\end{frame}
			
			%\subsubsection{Le 5 grands principes de l'évaluation au service de l'apprentissage \citep{black2004}} 
				%\begin{frame}
			 		%\frametitle{Le 5 grands principes de l'évaluation au service de l'apprentissage\citep{black2004}}
					%\begin {itemize}
						%\item Fournir aux élèves des appréciations efficaces
						%\item Une implication active des élèves dans leur propre apprentissage
						%\item L'adaptation de l'enseignement afin qu'il prenne en compte les résultats de l'évaluation
						%\item La prise en compte de l'influence considérable del'évaluation sur la motivation et l'estime de soi des élèves. Ces deux points ayant une influence critique sur les capacités d'apprentissage.
						%\item Le besoin qu'ont les élèves d'être en mesure de s'autoévaluer et de comprendre comment s'améliorer
					%\end{itemize}
				%\end{frame}
				
				%\subsubsection{Comment concrétiser les 5 grands principes de l'évaluation au service de l'apprentissage \citep{Missaoui2013}et \citep{itslearning2012a} }
				%\begin{frame}
			 		%\frametitle{Comment concrétiser les 5 grands principes de l'évaluation au service de l'apprentissage \citep{Missaoui2013} et \citep{itslearning2012a}}
					%\begin {itemize}
						%\item Négocier les objectifs d'apprentissage
						%\item Combiner la classe réelle à la classe virtuelle (quand c'est possible)
						%\item Utiliser un environnement numérique d'apprentissage pour :
							%\begin {itemize}
								%\item Dépôt de ressource
								%\item Activités avec des feed-back réguliers
								%\item 
							%\end{itemize}
						
					%\end{itemize}
				%\end{frame}
			
			
		\subsection{Les facteurs technologiques} 
			\begin{frame}
			 	\frametitle{Les facteurs technologiques \citep{NorthCarolina2013}}
				\begin {itemize}
							\item Déternir un \textbf{réseau développé}\,;
							\item Une couverture complète du WiFi est nécessaire \citep{Stansbury2013B}\,;
							\item Disposer d'une \textbf{large bande passante}\,;
							\item Posséder des \textbf{équipements informatiques performants}.						
				\end{itemize}
			\end{frame}
			
		\subsection{Les facteurs économiques} 
			\begin{frame}
			 	\frametitle{Les facteurs économiques\citep{NorthCarolina2013}}
				\begin {itemize}
					\item L'évaluation en ligne mène à des \textbf{réductions de coût au niveau national et local}\,;
					\item L'évaluation en ligne offre un \textbf{temps réponse plus rapide} pour les enseignants et les élèves / étudiants.
				\end{itemize}
			\end{frame}
			
		\subsection{Les facteurs sociaux} 
			\begin{frame}
			 	\frametitle{Les facteurs sociaux \citep{NorthCarolina2013}}
				\begin {itemize}
					\item \textbf{Amélioration de l'accessibilité} par l'usage d'accommodements pour l'élève ou l'étudiant (ex.: audio, vidéo, couleur d'arrière-plan alternative, etc.).
					
				\end{itemize}
			\end{frame}
			
			\subsection{Les facteurs liés aux processus et au contrôle de la qualité} 
			\begin{frame}
			 	\frametitle{Les facteurs liés aux processus et au contrôle de la qualité \citep{authority2014a}}
				
				\begin {itemize}
					\item Le processus doit démontrer la \textbf{cohérence} et de la \textbf{fiabilité}\,;
					\item Le personnel doit avoir les \textbf{compétences appropriées pour gérer et exécuter ces processus}\,;
					\item Les organisations doivent être \textbf{imputables en ce qui concerne la qualité de ces processus devant des organismes indépendants}\,;
					\item Une \textbf{certaine flexibilité doit être maintenue} pour que ces processus et compétences soient capables d'\textbf{évoluer en fonction de l'évolution technologique}.
					
				\end{itemize}
			\end{frame}
			
			
	\section{Pratiques} 
			
		\subsection{Québec : formation générale des jeunes (FGJ)} 
			\begin{frame}[allowframebreaks]
				  \frametitle{Québec : formation générale des jeunes (FGJ)}
				 \begin{description}[Second Item]
					\item[Primaire / 1\up{re} à la 3\up{e} secondaire]  \ \ \par Il existe \textbf{peu ou pas de politiques} au sujet de l'évaluation en ligne.
						\begin{itemize}
							\item 75\,\% de la note finale d'un élève est composé d'épreuves conçues par l'enseignant. L'enseignant \textbf{a la liberté de choisir le type d'épreuve} qu'il administre ainsi que la pondération. \textbf{Ces épreuves peuvent être administrées en ligne}\,;
							\item 25\,\% de la note finale d'un élève est composé d'épreuves ministérielles. Ces épreuves sont uniquement offertes en format papier, sauf pour les élèves à besoins particuliers. \textbf{Il est présentement interdit d'administrer ces épreuves en ligne}\,;
							\item Les enseignants utilisent les \textbf{questionnaires sur les environnements numériques d'apprentissage} ou \textbf{la remise de travaux}.
						\end{itemize}
					\framebreak
					\item[4\up{e} et 5\up{e} secondaire]  \ \ \par Il existe \textbf{peu ou pas de politiques} au sujet de l'évaluation en ligne.
						\begin{itemize}
							\item 50\,\% de la note finale d'un élève est composé d'épreuves conçues par l'enseignant. \textbf{L'enseignant a la liberté de choisir le type d'épreuve qu'il administre ainsi que la pondération}. Ces épreuves peuvent être administrées en ligne\,;
							\item 50\,\% de la note finale d'un élève est composé d'épreuves ministérielles. Ces épreuves s'effectuent sur papier, sauf pour les élèves à besoins particuliers. \textbf{Il est présentement interdit d'administrer ces épreuves en ligne. Ces épreuves ont préséance sur la note finale}\,;
							\item Les enseignants utilisent les \textbf{questionnaires sur les environnements numériques d'apprentissage} ou la \textbf{remise de travaux}.
						\end{itemize}
					\framebreak
					\item[Élèves à besoin particulier]  \ \ \par \textbf{Des balises techniques} pour la passation des épreuves ministérielles existent dans les commissions scolaires.
					\begin{itemize}
						\item Les adaptations sont \textbf{autorisées dans le cadre de la Gestion de la sanction des études}\,;
						\item Les épreuves peuvent être administrées sur ordinateur afin d'offrir de {l'aide à l’écriture et à
la lecture} pour la passation des épreuves ministérielles.\,;
						\item Les épreuves sont administrées à partir de clés USB (lecture et écriture)\,;
						\item L'élève peut utiliser son ordinateur personnel\,;
						\item \textbf{L'enseignant produit l'impression de la copie finale de l'épreuve}, l'élève signe chaque page de l'épreuve la clé USB de l'élève est conservée.
\end{itemize}.
					

\end{description}
			\end{frame}
			
			
	\subsection{Québec : formation professionnelle (FP)} 
			\begin{frame}[allowframebreaks]
				  \frametitle{Québec : formation professionnelle (FP)}
				 \begin{description}
					\item[Élèves réguliers] \ \ \par Il y a \textbf{peu ou pas de politiques} d'évaluation en ligne.
						\begin{itemize}
							\item Les épreuves ministérielles sont reçues en format papier\,;
							\item En formation professionnelle, le mode d’identification peut varier selon le type d’apprentissage (compétence de participation, compétence de comportement incluant une évaluation théorique ou pratique)\footnote{Melaçon, mai 2015. Tiré d'un document en réponse à une demande de la Direction de la sanction des études interpellée sur l'évaluation des compétences en formation professionnelle sans présence physique}\,; 
							\item La Direction de la sanction des études a été récemment interpellée sur l’évaluation des compétences sans présence physique de l’élève en salle d’examen (évaluation à l’aide d’Internet)\,;
							\item Dans certains programmes en alternance travail études : on utilise un \textbf{logiciel de contrôle à distance} et un \textbf{poste d’examen dédié} disponible à distance. Par exemple (tiré d'un entretien avec le Centre sectoriel des plastiques)\,: 
							\begin{itemize}
								\item à l'aide d'un logiciel qui permet le contrôle à distance des ordinateurs, l'enseignant se connecte au poste informatique de l'apprenant en milieu de travail,
								\item lorsque l'enseignant a le contrôle du poste de l'apprenant, il redirige cette connexion sur un poste d'examen situé au Centre de formation,
								\item l'apprenant a ainsi accès à cette connexion et à son évaluation qui a été réalisée en mode protégé sur un document Word et déposée. L'apprenant ne peut donc pas modifier l'évaluation, ni l'imprimer; il ne peut répondre qu'aux questions,
								\item l'enseignant peut aussi utiliser un questionnaire en mode protégé sur l'environnement numérique d'apprentissage TICFP (Camillo),
								\item l'élève est observé durant toute l'évaluation par une caméra (Webcam) et un microphone\,;
							\end{itemize}
							\framebreak
							\item Pour les compétences dont les évaluations ne sont pas ministérielles\,:
							 \begin{itemize}
								\item certains centres de formation utilisent un \textbf{logiciel de visioconférence pour l'évaluation en ligne}\,:
									\begin{itemize}
										\item l'élève s'identifie à l'aide d'une pièce d'identité avec photo,
										\item l'élève remplie et signe le cahier du candidat sur le tableau blanc du logiciel de visioconférence tout en prenant soin de faire une capture d'écran,
										\item le portfolio numérique est utilisé (souvent sous la forme de fichier Word ou PDF) pour l'évaluation de travaux pratiques,
										\item les éclatements pour différencier et personnaliser l'évaluation et séparer les élèves sont utilisés,
										\item le logiciel de visioconférence est utilisé afin de faire des mises en situation et des simulations (par exemple : Animation de réunion d'équipe),
										\item les élèves ont la permission de se filmer pour des évaluations pratiques différées qui sont déposées par Internet ou envoyer par courrier,
										\item \textbf{Tout est imprimé dans le dossier de l'élève}\,;
									\end{itemize}
								\end{itemize}
							\framebreak
							\item Dans le cadre des \textbf{stages}, plusieurs centres de formation utilisent des fichiers Word ou des formulaires en ligne pour le journal de bord. Par contre, tout est imprimé\,;
							\item Les \textbf{fiches de verdict et grilles d'évaluation} doivent toujours être imprimées\,;
							\item \textbf{Plusieurs centres de formation professionnelle et de commissions scolaires appliquent les mêmes règles que celles exigées pour les épreuves ministérielles aux épreuves non ministérielles}.
					\end{itemize}
					\item[Élèves à besoin particulier] \ \ \par Il y a \textbf{peu ou pas de balises techniques} pour la passation des épreuves ministérielles qui existent dans les commissions scolaires.
					\begin{itemize}
						\item Les \textbf{adaptations sont autorisées dans le cadre de la Gestion de la sanction des études}, notamment celles concernant les aides à l’écriture et à la lecture pour la passation des épreuves ministérielles.
						\item \textbf{Les épreuves ministérielles sont la plupart du temps reçues en format papier
						\item Administrer les épreuves de façon numérique n'est pas une pratique courante}.
					\end{itemize}
				\end{description}	
			\end{frame}
			
					
		\subsubsection{Prérequis en formation professionnelle} 
				\begin{frame}
					\frametitle{Prérequis (Melançon, 2015)}
				 	
				 	\begin {itemize}
						\item Vérification de l’identité : utilisation de la photographie de l’étudiant\,;
						\item Respect des règles d’administration\,:
							\begin {itemize}
								\item voir et parler avec la personne,
								\item valider l’identité de la personne,
								\item questionner le candidat afin de s’assurer que les travaux ont été effectué,
								\item se filmer et envoyer le tout avec une signature et une date.
							\end{itemize}
						\item Contrôler l’accès au matériel\,;
						\item Veiller à la validité du résultat.
					\end{itemize}
				\end{frame}
				
		
		
		\subsection{Dans le monde} 
						
			\subsubsection{Caroline du Nord} 
				\begin{frame}[allowframebreaks]
					  \frametitle{Caroline du Nord \citep{NorthCarolina2013}}
				 	\begin {itemize}
						\item 2005 : premiers tests en ligne\,;
						\item 2006-2007 :  l'examen EOC  (End-of-course) en physique était disponible en ligne\,;
						\item 2007-2008 les examens EOC  (End-of-course) de toutes les disciplines sont disponibles en ligne\,;
						\item 2011-2012, \textbf{19\,\% des examens sont administrés en ligne}.
					\end{itemize}
					\par Le NCDPI a amorcé une réforme numérique afin d'offrir un accès aux examens à travers l'état, de former des élèves / étudiants compétitifs et de développer des pratiques technopédagogiques abordables et pérennes. 
					\par Ils ont publié un guide des bonnes pratiques.
					\framebreak
					\par Ce guide qui répond aux questions sur :
					\begin {itemize}
						\item le calendrier des épreuves,
						\item la planification financière,
						\item les préalables techniques afin d'effectuer la transition vers le numérique,
						\item propose des études de cas.
						
					\end{itemize}
					\par 
				\end{frame}
		
		
			\subsubsection{Le reste des États-Unis} 
				\begin{frame}[allowframebreaks]
					  \frametitle{Le reste des États-Unis \citep{NorthCarolina2013}}
				 	\begin {itemize}
						\item Plusieurs états ont déjà un système d'évaluation en ligne\,:
							\begin {itemize}
								\item Virginie (\textbf{64\,\% de toutes les évaluations sont administrés en ligne}),
								\item Orégon (\textbf{en 2007-2008, les évaluations doivent toutes être administré en ligne})\,;
							\end{itemize}
						\framebreak
						\item D'autres états sont en transition vers l'évaluation en ligne simultanément à l'adoption de \textit{Common Core State Standards} (42 des 48 états ont adopté le Common Core)\,;
						\item \textbf{17 états sont membres du \textit{Smarter Balanced Assessment Consortium} qui a créé un système d'évaluation en ligne qui est arrimé au \textit{Common Core State Strandards}} (CCSS). Ce consortium est situé à l'UCLA’s Graduate School of Education\,\&\,Information Studies (GSE\,\&\,IS).
					\end{itemize}					
				\end{frame}
				
			\subsubsection{Royaume-Uni} 
				\begin{frame}
					  \frametitle{Royaume-Uni \citep{authority2014a}}
				 	\begin {itemize}
						\item \textbf{\textit{British Standards Institution code of practice for the use of information technology (IT) in the Delivery if assessments}\footnote{BS 7988:2002}. Ce code est devenu un standard international\footnote{ISO/IEC 23988:2007}}\,;
						\item Une grande place est accordée au \textbf{portfolio numérique}\,;
						\item Une \textbf{certification créditée pour le personnel impliqué} dans la prestation de service d'évaluation en ligne existe.
					\end{itemize}					
				\end{frame}
				
			\subsection{Documents de référence importants} 
				\begin{frame}[allowframebreaks]
					  \frametitle{Documents de référence importants}
				 		\begin{description}[Second Item]
							\item[Québec] Les pratiques et défis de l'évaluation en ligne \citep{audet2011a}
							\item[États-Unis] Online Assessments Best Practices Guide \citep{NorthCarolina2013}
							\item[Angleterre] Effective Practive with e-Assessment. An overview of technologies, policies and practices in further and higher education \citep{jisc2007a}
							\item[Écosse] E-assesment. Guide to effective practice \citep{authority2014a}
							\framebreak
							\item[ISO] Revised comments on e-Assessment \citep{ISO2008a}
					\end{description}
				\end{frame}
				
	%\section{La question du plagiat} 
		%\begin{frame}
			 % \frametitle{}
			% \begin {itemize}
				% \item 
				% \item 
			%\end{itemize}
		%\end{frame}
		
	\section{Cadre de référence} 
		%\begin{frame}
			 % \frametitle{}
			% \begin {itemize}
				% \item 
				% \item 
			%\end{itemize}
		%\end{frame}
		
		\subsection{Une nécessité économique} 
			\begin{frame}[allowframebreaks]
			 	\frametitle{Une nécessité économique \citep{Dubreucq2011}}
				\begin {itemize}
					\item Plusieurs institutions académiques adoptent les évaluations en ligne\,;
					\item Les environnements numériques d'apprentissage intègrent des dispositifs d'évaluation en ligne\,;
					\item Les organismes vivent une grande pression (uniformisation)\,;
					\item L'uniformisation crée des tensions\,;
					\item Les exigences d'un bon enseignement en ligne restent à être mieux identifiées.
					\framebreak
					\item Il faut également tenir compte des pressions pour l'amélioration de la productivité soulevées par \citet{audet2011a}, \citet{dirks1998a} et \citet{becta2006a} : 
						\begin {itemize}
							\item les budgets sont limités,
							\item le nombre d'étudiants est en hausse,
							\item les besoins augmentent,
							\item l'évaluation prend \textbf{35\,\% du temps de l'enseignant expérimenté et 56\,\% du temps des nouveaux enseignants}.
						\end{itemize}		
				\end{itemize}
				\framebreak
				La plupart des systèmes éducatifs convergent vers l'évaluation en ligne.
				\citet{Dubreucq2011}, déclare : «\,Qui touche aux évaluations touche à l'ensemble d'un système d'enseignement et d'apprentissage\,».
			\end{frame}
			
		
			
			\subsection{Des outils de planification} 
				\begin{frame}
			 		\frametitle{Des outils de planification \citep{Dubreucq2011}}
					\textbf{Un cadre de référence sur l'évaluation en ligne donnerait des outils aux professionnels de l'éducation pour bien planifier l'évaluation en ligne}  et pour faciliter :
					\begin {itemize}
						\item la formulation des objectifs de la formation,
						\item l'identification des types de compétences à développer.
					\end{itemize}
					Ces éléments sont au c\oe ur de la formation.
				\end{frame}
			
			\subsection{Des outils de diversification} 
				\begin{frame}
			 		\frametitle{Des outils de diversification\citep{Dubreucq2011}}
			 		\textbf{Un cadre de référence sur l'évaluation en ligne permettrait aux professionnels de l'éducation d'utiliser l'évaluation en ligne}\,: 
					\begin {itemize}
						\item pour des évaluations plus larges,
						\item pour la mise en oeuvre de compétences diversifiées (ex.: créativité, collaboration),
						\item pour la réflexion.
					\end{itemize}
				\end{frame}
				
			\subsection{Des outils de rétroaction} 
				\begin{frame}
			 		\frametitle{Des outils de rétroaction\citep{Dubreucq2011}}
					\textbf{Un cadre de référence sur l'évaluation en ligne permettrait aux professionnels de l'éducation}\,: 
					\begin {itemize}
						\item de \textbf  {normaliser} les corrections,
						\item d'\textbf {individualiser} les corrections,
						\item d'\textbf{offrir une plus grande automie à l'élève ou l'étudiant},
						\item d'\textbf{économiser du temps} sur les évaluations individuelles.
					\end{itemize}
				\end{frame}
				
			\subsection{Une réflexion sur la fraude} 
				\begin{frame}[allowframebreaks]
			 		\frametitle{Une réflexion sur la fraude \citep{Dubreucq2011}}
			 		Un cadre de référence sur l'évaluation en ligne permettrait de donner au professionnel de l'éducation des lignes directrices\,: 
					\begin {itemize}
						\item en décrivant les \textbf{causes}, \textbf{l'évolution} et la \textbf{croissance du phénomène}, particulièrement en ligne,
						\item en \textbf{développant les compétences informationnelles} à ce sujet,
						\item en donnant les \textbf{outils aux professionnels de l'éducation} afin de permettre\citep{bergadaa2008relation}\,:
							\begin {itemize}
								\item \textbf{d'améliorer les compétences en recherche documentaire},
								\item \textbf{de développer un sens critique dans le traitement de l'information}\,;
							\end{itemize}
						\item en repensant l'évaluation comme un moyen de mesurer des habiletés \textbf{de plus haut niveau que la simple mémorisation de savoirs}. C'est-à-dire :
							\begin {itemize}
								\item l'analyse,
								\item la synthèse, 
								\item la confrontation de documents.
							\end{itemize}
						\item en favorisant des \textbf{évaluations plus variées et plus continues}.
								
					\end{itemize}
				\end{frame}
			


%\section{Bibliographie}
%\subsection{Bibliographie}
\frame[allowframebreaks]{\frametitle{Bibliographie}

\bibliographystyle{apalike}
\bibliography{bibliographie} %bibtex file name without .bib extension
}
\framebreak
\par L’intention de ce document est de respecter pleinement les droits des créateurs des ressources
utilisées.
	\par En ce qui concerne les citations insérées selon le principe de l'utilisation équitable, veuillez les contacter ou respecter les droits d’utilisation précisés dans les documents d’origine avant de les réutiliser.
	\par Si vous estimez que certains éléments de ce rapport ne respectent pas intégralement les droits de vos
publications, veuillez nous en aviser afin que les modifications nécessaires puissent être apportées au : \url{mailto:sprudhomme@cslaval.qc.ca}.
	\par Cette \oe uvre, création, site ou texte est sous licence Creative Commons Attribution\,-\,Pas d’Utilisation Commerciale\,-\,Partage dans les Mêmes Conditions 4.0 International. \\
	\par 
	 Pour accéder à une copie de cette licence, merci de vous rendre à l'adresse suivante\,: \url{http://creativecommons.org/licenses/by-nc-sa/4.0/} ou envoyez un courrier à 

\par Creative Commons, 444 Castro Street, Suite 900, Mountain View, California, 94041, USA.
\par Ce document a été réalisé en \LaTeX, avec l'environnement Beamer. Vous pouvez trouver le code source ici : \url{https://goo.gl/NeExOf}. Vous pouvez avoir accès à cette présentation ainsi qu'à d'autres ressources sur\url{https://goo.gl/jvzi0s}
%\begin{pspicture}(1in,1in)
    %\psbarcode{test string}{}{qrcode}
  %\end{pspicture}
  

\end{document}

