% Auteur : Steve Prud’Homme
% Cette oeuvre, création, site ou texte est sous licence Creative Commons Attribution - Pas d’Utilisation Commerciale - Partage dans les Mêmes Conditions 4.0 International. Pour accéder à une copie de cette licence, merci de vous rendre à l'adresse suivante 
% http://creativecommons.org/licenses/by-nc-sa/4.0/ ou envoyez un courrier à 
% Creative Commons, 444 Castro Street, Suite 900, Mountain View, California, 94041, USA.
\documentclass[aspectratio=169]{beamer}
\usepackage{color}
\usepackage{beamerthemesplit} % new 
\usepackage[french]{babel}
\usepackage[utf8]{inputenc}
\usepackage{tikz}
\usepackage[fixlanguage]{babelbib}
\selectbiblanguage{french}
% Natlib pour la bibliographie
\usepackage{natbib}
\usepackage{url}
\usetikzlibrary{mindmap,shadows,shapes,backgrounds}
\usepackage[T1]{fontenc}
\setbeamertemplate{bibliography item}[text]
\usepackage{multicol}


\definecolor{MightySlate}{RGB}{85,98,112}
\definecolor{Pacifica}{RGB}{78,205,196}
\definecolor{AppleChic}{RGB}{199,244,100}
\definecolor{CheeryPink}{RGB}{255,107,107}

\setbeamercolor{titlelike}{parent=structure}
\setbeamerfont*{title}{size=\huge}
\setbeamercolor{title}{bg=MightySlate, fg=white}
\setbeamercolor{author}{bg=Pacifica, fg=white}
\setbeamercolor{institute}{bg=AppleChic, fg=black}
\setbeamercolor{date}{bg=CheeryPink, fg=white}

\definecolor{DTUred}{RGB}{178,20,20}
\setbeamercolor*{palette primary}{use=structure,fg=white,bg=MightySlate}


\begin{document}
	\title{L'évaluation en ligne} 
	\author{Steve Prud'Homme} 
	\institute{GTN-Québec - Commission scolaire de Laval} 
	\date{\today} 

	
	\usebackgroundtemplate{%
  \includegraphics[width=\paperwidth,height=\paperheight]{sommaire.png}} 
	
	\frame{\titlepage} 
	\usebackgroundtemplate{ } 
	\section{Sommaire} 
		\begin{frame}
			Cette présentation vise à : 
			\frametitle{Sommaire}
			\begin {itemize}
				\item Effecturer un court état des lieux de l'évaluation en ligne et de ses pratiques en formation professionnelle
				\item Familiariser l'auditoire avec l'évaluation en ligne 
				\item Démontrer qu'il est pertinent d'adopter des pratiques harmonisées en ce qui concerne l'évaluation en ligne
				\item Promouvoir l'idée qu'il est essentiel que le MEES établisse des lignes directrices rapidement en ce qui concerne l'évaluation en ligne

			\end{itemize}
		\end{frame}
	\frame[allowframebreaks]{\frametitle{Ordre du jour}\tableofcontents}


	\section{Contexte} 
		\begin{frame}[allowframebreaks]
			\frametitle{Le contexte de la formation à distance et en ligne}
			Le \citet{OLC2015a}, conclus dans une grande étude étatsunienne qu'en 2015\footnote{Infographie : \url {http://www.onlinelearningsurvey.com/reports/2015SurveyInfo.pdf} } :
			\begin {itemize}
				\item Il y a eu une hausse d'inscription en formation à distance de 3,9 \% par rapport à l'année dernière
				\item Plus du quart des étudiants (28 \%) prend des cours en formation à distance (un  total de 5 828 826 étudiants, une hausse annuelle de 217,275)  pour une population de 321 773 631 (soit 1,8  \%).
				\item Le pourcentage de directeurs universitaires qui considèrent les résultats de l'apprentissage dans l'éducation en ligne comme étant égaux ou supérieurs à ceux de l'enseignement en face à face est désormais de 71,4 \%.
				\item Seuls 29,1 \% des dirigeants universitaires rapportent que leur faculté reconnaît la « valeur et la légitimité de l'éducation en ligne ». Parmi les écoles ayant les inscriptions les plus lointaines, 60,1 \% rapportent la reconnaissance de la formation en ligne par la faculté tandis que seules 11,6 \% des écoles avec des inscriptions locales le font. 
			\end{itemize}
			\framebreak
			Selon la \citet{sofad2015a} en 2014 : 
			\begin {itemize}
				\item Le total des inscriptions à des cours à distance (formations générales et professionnelles) a augmenté de 3,6 \% et atteint un nouveau sommet à 56 608.
				\item Le nombre d’élèves concernés par ces cours a augmenté quant à lui de 3 \% pour atteindre un nouveau sommet à 29 386 pour une population, selon \citet{is2015a}, de 8 263 600 (soit 0,36 \%)
				\item La croissance est particulièrement forte en formation professionnelle à distance avec des augmentations de 8 \% du nombre d’inscriptions et de 10 \% pour le nombre d’élèves, pour atteindre de nouveaux sommets dans un cas comme dans l’autre (10 209 et 3 374)
			\end{itemize}
		\end{frame}
		
		\subsection{La question de l'évaluation en ligne} 
		\begin{frame}
			\frametitle{La question l'évaluation en ligne}
			\begin {itemize}
				\item Très peu discutée, selon \citet{audet2011a} qui cite \citet{buzzetto2006a} : « si le terme fait maintenant partie du vocabulaire courant de la formation, le concept d'évaluation électronique ou en ligne est encore relativement peu répandu »
				\item Il existe des désaccords à ce sujet
				
			\end{itemize}
		\end{frame}
		
	\subsection{L'aspect séducteur} 
		\begin{frame}
			\frametitle{L'aspect séducteur de l'évaluation en ligne\citep{Lamontagne2013}}
			\begin {itemize}
				\item Gestion facilitée
				\item Économie
				\item Standardisation
				\item Personnalisation
				\item Contrôle
				\item Rapidité
				\item Flexibilité
				\item Qualité
			\end{itemize}
		\end{frame}
	
	\subsection{Résultats} 
		\begin{frame}
			\frametitle{Des résultats ?}
			\citet {Stansbury2013C}, déclare en citant Scoot Smith, directeur TI à la Mooresville Graded School District (N.C), comme nous le rapporte \citet{Lamontagne2013} que :
			\begin {itemize}
				\item Les résultats aux tests d'état finaux connaissent des augmentations significatives (+20 \% sur 6 ans)
				\item Un taux de poursuite aux études supérieures de + 40 \%.
				\item Une augmentation de 16 \% dans la réussite globale
			\end{itemize}
			Selon \citet{Lamontagne2013}, ces améliorations sont dues à la systématisation des suivis et de l'attention accordée à chacun.
			\par \alert{Un changement induit par l'utilisation de la formation en ligne}
		\end{frame}
	
	\section{Facteurs} 
		
		\subsection{Les facteurs liés à la formation ou à l'aisance TIC} 
			\begin{frame}[allowframebreaks]
			 	\frametitle{Les facteurs liés à la formation ou à l'aisance TIC \citep{NorthCarolina2013} et \citep{Lamontagne2013}}
				\begin {itemize}
					\item Préparer le personnel enseignant et administratif
						\begin {itemize}
							\item Comprend la formation des responsables TI
							\item Comprend la formation des administrateurs et gestionnaires de tests à tous les niveaux (local, régional, provincial)
						\end{itemize}	
					\item Préparer les élèves ou étudiants
						\begin {itemize}
							\item Former les étudiants aux outils.
						\end{itemize}
					\item Selon \citet{Stansbury2013A} : « \textit{Without a basic technical foundation and statewide support, you can kiss online assessments goodbye.}». \citet{Stansbury2013A} cite Patches Hill, directeur des TI du Delaware Indiant River School District (IRSD) : « \textit{You need your state-level infrastructure implementation, state assessment implementation [summative standardization], and personnalized learning [formative statardization]. This is a multistep process and it can't be done instantly.}»
					
				\end{itemize}
			\end{frame}
			
		\subsection{Les facteurs pédagogiques} 
			\begin{frame}[allowframebreaks]
			 	\frametitle{Les facteurs pédagogiques \citet{audet2011a}}
				\begin {itemize}
					\item Les objectifs d'apprentissage
						\begin {itemize}
							\item Les outils du Web facilitent l'identification de ces objectifs, leur mise en relation et leur partage entre concepteurs
							\item Ils soutiennent leur diffusion aux étudiants et clarifient ainsi les attentes de l'évaluation
						\end{itemize}
					\item Les compétences évaluées
						\begin {itemize}
							\item Le Web rend justement beaucoup plus simple la réalisation d'activités d'évaluation visant la démonstration de compétences diversifiées, incluant par exemple la créativité ou la collaboration
							\item On vise des compétences de plus haut niveau que la simple mémorisation 
						\end{itemize}
					\item La finalité des évaluations
						\begin {itemize}
							\item L'évaluation par internet contribue de façon importante à cette tendance vers une évaluation plus formative et plus formatrice.
							\item Elle facilite également l'administration des tests diagnostiques et l'utilisation à plusieurs fins d'un même ensemble de contenus d'évaluation.					
						\end{itemize}
					\item La rétroaction
						\begin {itemize}
							\item Les outils liés à l'évaluation par Internet facilitent et accélèrent la transmission de la rétroaction
							\item Selon \citet{whitelock2006a}, cité par \citet{audet2011a}, cette rapidité de rétroaction est déterminante dans le développement de l'évaluation en ligne, car elle permet :
								\begin {itemize}
									\item L'automatisation
									\item La remise immédiate à l'apprenant
									\item La réutilisation
								\end{itemize}
							\item Elle permet aussi de rendre l'évaluation plus captivante parce qu'elle permet d'y insérer des éléments multimédias ou des hyperliens.
						\end{itemize}
					\framebreak 
					\item Les activités réalisées
						\begin {itemize}
							\item La réalisation d'activité d'évaluation :
								\begin {itemize}
									\item Plus variées
									\item Plus authentiques
									\item Plus collaboratives
								\end{itemize}
							\item Presque que toutes les formes de démonstrations de compétences (exemple : les wikis)
							\item Il est facile de combiner des évaluations alternatives avec des tests formatifs ou sommatifs automatisés (diversité versus tâche de l'enseignant)										
						\end{itemize}
					\framebreak 
					\item Fréquence de la mesure
						\begin {itemize}
							\item Selon \citet{audet2011a} : « Le Web vient alors supporter la compilation et l'analyse des divers résultats obtenus »
						\end{itemize}
					\framebreak 
					\item Les évaluateurs
						\begin {itemize}
							\item Les outils d'évaluation en ligne facilitent particulièrement les formes alternatives d'évaluation en permettant :
								\begin {itemize}
									\item L'anonymat
									\item La pondération complexe
									\item La compilation des résultats
									\item La discussion
									\item La négociation
									\item La conservation des traces
								\end{itemize}
						\end{itemize}
					\framebreak
					\item Les critères d'évaluation
						\begin {itemize}		
							\item L'utilisation de critères détaillés facilite par la compilation efficace de résultats
							\item L'élimination des contraintes liées à la diffusion des documents (diffusion des critères préalables et des résultats liés)
						\end{itemize}
					\item La portée. Selon \citet{jisc2007a}, cité par  \citet{audet2011a} :
						\begin {itemize}		
							\item  L'évaluation en ligne est souvent restreinte aux évaluations à portée faible\footnote{Habituellement formative et ses résultats demeurent locaux} et moyenne\footnote{Peuvent avoir des résultats locaux ou nationaux, mais ceux-ci ne sont pas déterminants dans la vie de la personne évaluée}.
							\item De plus en plus d'organisations l'envisagent dans des contextes de certification professionnelle, où l'impact d'un succès ou d'un échec est très important.
						\end{itemize}
					\item Selon \citet{audet2011a}, nous allons vers un alignement :
						\begin {itemize}		
							\item Selon \citet{astin1992a}, cité par \citet{audet2011a}, : « L'évaluation des la plus efficace quand elle reflète une compréhension de l'apprentissage comme multidimensionnel, intégrée et démontrée par une performance dans le temps», qui se traduit selon \citet{angelo1996a}, en quatre composante :
								\begin {itemize}
									\item La multiplicité des méthodes
									\item Les dimensions de l'apprentissage
									\item Les évaluateurs
									\item Les moments d'évaluation
								\end{itemize}
								\citet{audet2011a}, déclare : « L'évaluation en ligne facilite la mise en place des ces nombreux paramètres et le suivi qu'ils nécessitent », en fonction de l'intention pédagogique du concepteur.
						\end{itemize}
				\end{itemize}
			\end{frame}
			
			\subsubsection{Le 5 grands principes de l'évaluation au service de l'apprentissage \citep{black2004}} 
				\begin{frame}
			 		\frametitle{Le 5 grands principes de l'évaluation au service de l'apprentissage\citep{black2004}}
					\begin {itemize}
						\item Fournir aux élèves des appréciations efficaces
						\item Une implication active des élèves dans leur propre apprentissage
						\item L'adaptation de l'enseignement afin qu'il prenne en compte les résultats de l'évaluation
						\item La prise en compte de l'influence considérable del'évaluation sur la motivation et l'estime de soi des élèves. Ces deux points ayant une influence critique sur les capacités d'apprentissage.
						\item Le besoin qu'ont les élèves d'être en mesure de s'autoévaluer et de comprendre comment s'améliorer
					\end{itemize}
				\end{frame}
				
				\subsubsection{Comment concrétiser les 5 grands principes de l'évaluation au service de l'apprentissage \citep{Missaoui2013}et \citep{itslearning2012a} }
				\begin{frame}
			 		\frametitle{Comment concrétiser les 5 grands principes de l'évaluation au service de l'apprentissage \citep{Missaoui2013} et \citep{itslearning2012a}}
					\begin {itemize}
						\item Négocier les objectifs d'apprentissage
						\item Combiner la classe réelle à la classe virtuelle (quand c'est possible)
						\item Utiliser un environnement numérique d'apprentissage pour :
							\begin {itemize}
								\item Dépôt de ressource
								\item Activités avec des feed-back réguliers
								\item 
							\end{itemize}
						
					\end{itemize}
				\end{frame}
			
			
		\subsection{Les facteurs technologiques} 
			\begin{frame}
			 	\frametitle{Les facteurs technologiques}
				\begin {itemize}
					\item Selon la \citet{NorthCarolina2013}, citée par \citet{Lamontagne2013} il faut :
						\begin {itemize}
							\item Déternir un réseau développé. Selon \citet{Stansbury2013B}, une couverture complète du WiFi est nécessaire.
							\item Diposer d'une large bande passante
							\item Posséder des équipements informatiques performants
						\end{itemize} 
				\end{itemize}
			\end{frame}
			
		\subsection{Les facteurs économiques} 
			\begin{frame}
			 	\frametitle{Les facteurs économiques}
				Selon la \citet{NorthCarolina2013} :
				\begin {itemize}
					\item L'évaluation en ligne mène à des réductions de coût au niveau national et local
					\item L'évaluation en ligne offre un temps réponse plus rapide pour les enseignants et les élèves / étudiants
				\end{itemize}
			\end{frame}
			
		\subsection{Les facteurs sociaux} 
			\begin{frame}
			 	\frametitle{Les facteurs sociaux}
				\begin {itemize}
					\item Selon le \citet{NorthCarolina2013}, il y a une amélioration de l'accessibilité par l'usage d'accommodements pour l'élève ou l'étudiant (ex.: audio, vidéo, couleur arrière-plan alternative,etc.) 
					
				\end{itemize}
			\end{frame}
			
			\subsection{Les facteurs liés aux processus et au contrôle de la qualité \citep{authority2014a}} 
			\begin{frame}
			 	\frametitle{Les facteurs liés aux processus et au contrôle de la qualités}
				Selon la \citet{authority2014a} :
				\begin {itemize}
					\item Le processus doit démontré la cohérence et être fiable
					\item Le personnel doit avoir les compétences appropriées pour gérer et exécuter ces processus.
					\item Les organisations doivent être imputables en ce qui concerne la qualité de ces processus devant des organismes indépendants
					\item Une certaine flexibilité doit être maintenue pour que ces processus et compétences soient capables d'évoluer en fonction des améliorations technologiques .
					
				\end{itemize}
			\end{frame}
			
			
	\section{Pratiques} 
			
		\subsection{Au Québec en formation professionnelle} 
			\begin{frame}
				  \frametitle{Au Québec en formation professionnelle}
				  \begin {itemize}
					\item Melançon (2015), déclare qu'en formation professionnelle, le mode d’identification peut varier selon le type d’apprentissage (compétence de participation, compétence de comportement incluant une évaluation théorique ou pratique). 
					\item La Direction de la sanction des études a été récemment interpellée sur l’évaluation des compétences sans présence physique de l’élève en salle d’examen (évaluation à l’aide d’Internet)
				\end{itemize}
			\end{frame}
			
		\subsubsection{Prérequis} 
				\begin{frame}
					\frametitle{Prérequis}
				 	
				 	\begin {itemize}
						\item Vérification de l’identité : utilisation de la photographie de l’étudiant.
						\item Respect des règles d’administration
							\begin {itemize}
								\item Voir et parler avec la personne
								\item S’assurer de l’identité de la personne
								\item Questionner le candidat afin de s’assurer que les travaux ont été effectué.
								\item Se filmer et envoyer le tout avec une signature et une date.
							\end{itemize}
						\item Contrôler l’accès au matériel
						\item Veiller à la validité du résultat
					\end{itemize}
				\end{frame}
				
		\subsubsection{Centre sectoriel des plastiques} 
				\begin{frame}[allowframebreaks]
					\frametitle{Centre sectoriel des plastiques}
				 	\includegraphics [height=4cm]{plasturgie.png}				 	
				 	\framebreak
				 	\par Utilisation d’un logiciel de contrôle à distance et d’un poste d’examen dédié disponible à distance
				 	
				 	\begin {itemize}
						\item Utilisation de la plateforme TICFP (CAMILLO) pour l’assignation des horaires d’examen.
						\item L’enseignant se connecte au poste de l’apprenant (situé sur le lieu de travaille en ATE) à l’aide du logiciel Teamviewer
						\item Lorsque l’enseignant a le contrôle du poste de l’apprenant, il se connecte sur un poste d’examen situé au Centre sectoriel des plastiques.
						\item Sur cette ordinateur il y a un fichier Microsoft Word en mode formulaire protégé que l’élève rempli.
						\item L’utilisation d’un questionnaire sur la plateforme TICFP (CAMILLIO) est aussi possible sur le poste d’examen.
						\item Utilisation Webcam et micro.
					\end{itemize}
					
					
				\end{frame}
		
		\subsection{Dans le monde} 
						
			\subsubsection{Caroline du Nord} 
				\begin{frame}[allowframebreaks]
					  \frametitle{Caroline du Nord\citep{NorthCarolina2013}}
				 	\begin {itemize}
						\item 2005 : premiers tests en ligne
						\item 2006-2007 :  l'examen EOC  (End-of-course) en physique était disponible en ligne
						\item 2007-2008 les examens EOC  (End-of-course) de toutes les disciplines sont disponibles en ligne
						\item 2011-2012, 19 \% des examens sont administrés en ligne
					\end{itemize}
					\par Le NCDPI a amorcé une réforme numérique afin d'offrir un accès aux examens à travers l'état, de produire des élèves / étudiants compétitifs et de développer des pratiques technopédagogiques abordables et pérennes. 
					\par Ils ont publié un guide des bonnes pratiques.
					\par Ce guide répond aux questions sur :
					\begin {itemize}
						\item Le calendrier des épreuves
						\item La planification financière
						\item Les préalables techniques afin d'effectuer la transition vers le numérique
						
						
					\end{itemize}
					\par Il propose des études de cas
				\end{frame}
		
		
			\subsubsection{Le reste des États-Unis} 
				\begin{frame}
					  \frametitle{Le reste des États-Unis \citep{NorthCarolina2013}}
				 	\begin {itemize}
						\item Plusieurs états ont déjà un système d'évaluation en ligne d'état
							\begin {itemize}
								\item Virginie (en 64 \% de toutes les évaluations sont administrés en ligne)
								\item Orégon
							\end{itemize}
						\item D'autres états sont en transition vers l'évaluation en ligne simultanément à l'adoption de \textit{Common Core State Standards}
						\item 17 états sont membre du \textit{Smarter Balanced Assessment Consortium} qui a créé un système d'évaluation en ligne qui est arrimé au \textit{Common Core State Strandards} (CCSS). Ce consortium est situé à l'UCLA’s Graduate School of Education \& Information Studies (GSE\&IS)
					\end{itemize}					
				\end{frame}
				
			\subsubsection{Royaume-Uni} 
				\begin{frame}
					  \frametitle{Royaume-Uni \citep{authority2014a}}
				 	\begin {itemize}
						\item \textit{British Standards Institution code of practice for the use of information technology (IT) in the Delivery if assessments}\footnote{BS 7988:2002}. Ce code est devenu un standard international\footnote{ISO/IEC 23988:2007}
						\item Une grande place est accordée au portfolio numérique.
						\item Une certification créditée pour le personnel impliqué, prestation de service d'évaluation en ligne existe.
						\item 
					\end{itemize}					
				\end{frame}
				
			\subsection{Documents de référence importants} 
				\begin{frame}[allowframebreaks]
					  \frametitle{Documents de référence importants}
				 		\begin{description}[Second Item]
							\item[Québec] Les pratiques et défis de l'évaluation en ligne \citep{audet2011a}
							\item[États-Unis] Online Assessments Best Practices Guide \citep{NorthCarolina2013}
							\item[Angleterre] Effective Practive with e-Assessment. An overview of technologies, policies and pactice in further and higher education \citep{jisc2007a}
							\item[Écosse] E-assesment. Guide to effective practice \citep{authority2014a}
							\framebreak
							\item[ISO] Revised comments on e-Assessment \citep{ISO2008a}
					\end{description}
				\end{frame}
				
	\section{La question du plagiat} 
		\begin{frame}
			 % \frametitle{}
			% \begin {itemize}
				% \item 
				% \item 
			%\end{itemize}
		\end{frame}
		
	\section{Vers un cadre de référence national} 
		\begin{frame}
			 % \frametitle{}
			% \begin {itemize}
				% \item 
				% \item 
			%\end{itemize}
		\end{frame}
		
		\subsection{Une nécessité économique \citep{Dubreucq2011}} 
			\begin{frame}[allowframebreaks]
			 	\frametitle{Une nécessité économique \citep{Dubreucq2011}}
				\begin {itemize}
					\item Plusieurs institutions adoptent les évaluations en ligne
					\item Les environnements numériques d'apprentissage intègrent des dispositifs d'évaluation en ligne
					\item Il y a donc une grande pression sur les organismes (uniformisation)
					\item L'uniformisation crée des tensions et sont loin des exigences d'un bon enseignement.
					\framebreak
					\item Selon plusieurs auteurs \footnote{\citet{audet2011a}, \citet{dirks1998a} et \citet{becta2006a}}, il faut également tenir compte des pressions pour l'amélioration de la productivité. 
						\begin {itemize}
							\item Les budgets sont limités
							\item Le nombre d'étudiants en hausse
							\item Les besoins augmentent
							\item L'évaluation prend 35 \% du temps de l'enseignant expérimenté pour 56 \% du temps pour les nouveaux enseignants
						\end{itemize}		
				\end{itemize}
				\framebreak
				La plupart des systèmes éducatifs convergent vers l'évaluation en ligne.
				\citet{Dubreucq2011}, déclare : « Qui touche aux évaluations touche à l'ensemble d'un système d'enseignement et d'apprentissage ».
			\end{frame}
			
		\subsection{Des outils pour les professionnels de l'éducation\citep{Dubreucq2011}} 
			\begin{frame}
			 	% \frametitle{Des outils pour les professionnels de l'éducation\citep{Dubreucq2011}}
				% \begin {itemize}
					% \item 
					% \item 
				%\end{itemize}
			\end{frame}
			
			\subsubsection{Des outils de planification \citep{Dubreucq2011}} 
				\begin{frame}
			 		\frametitle{Des outils de planification \citep{Dubreucq2011}}
					Un cadre de référence sur l'évaluation en ligne donnerait des outils aux professionnels de l'éducation pour bien planifier l'évaluation en ligne qui facilite :
					\begin {itemize}
						\item La formulation des objectifs de la formation
						\item L'identification des types de compétences à développer
					\end{itemize}
					Ces éléments sont au coeur de la formation.
				\end{frame}
			
			\subsubsection{Des outils de diversification\citep{Dubreucq2011}} 
				\begin{frame}
			 		\frametitle{Des outils de diversification\citep{Dubreucq2011}}
			 		Un cadre de référence sur l'évaluation en ligne permettrait aux professionnels de l'éducation d'utiliser l'évaluation en ligne qui permet : 
					\begin {itemize}
						\item Des évaluations plus larges
						\item La mise en oeuvre de compétences diversifiées (ex.: créativité, collaboration)
						\item La réflexion
					\end{itemize}
				\end{frame}
				
			\subsubsection{Des outils de rétroaction\citep{Dubreucq2011}} 
				\begin{frame}
			 		\frametitle{Des outils de rétroaction\citep{Dubreucq2011}}
					Un cadre de référence sur l'évaluation en ligne permettrait aux professionnels de l'éducation de : 
					\begin {itemize}
						\item Normaliser les corrections
						\item Individualiser les corrections
						\item Offrir une plus grande automie à l'élève ou l'étudiant
						\item D'économiser du temps sur les évaluations individuelles
					\end{itemize}
				\end{frame}
				
			\subsubsection{Une réflexion sur la fraude \citep{Dubreucq2011}} 
				\begin{frame}[allowframebreaks]
			 		\frametitle{Une réflexion sur la fraude \citep{Dubreucq2011}}
			 		Un cadre de référence sur l'évaluation en ligne permettrait de donner au professionnel de l'éducation des lignes directrices en : 
					\begin {itemize}
						\item Décrivant les causes, l'évolution et la croissance du phénomène, particulièrement en ligne.
						\item Développant les compétences informationnelles à ce sujet.
						\item Donnant selon \citet{bergadaa2008relation}, les outils aux professionnels de l'éducation afin de permettre  :
							\begin {itemize}
								\item d'améliorer les compétences en recherche documentaire
								\item de développer un sens critique dans le traitement de l'information 
							\end{itemize}
						\item Repenser l'évaluation comme un moyen de mesurer des habiletés de plus haut niveau que la simple mémorisation de savoirs. C'est-à-dire :
							\begin {itemize}
								\item L'analyse
								\item La synthèse 
								\item La confrontation de documents
							\end{itemize}
						\item Favoriser les évaluations plus variées et plus continues
								
					\end{itemize}
				\end{frame}
			


\section{Bibliographie}
\subsection{Bibliographie}
\frame[allowframebreaks]{\frametitle{Bibliographie}

\bibliographystyle{apalike}
\bibliography{bibliographie} %bibtex file name without .bib extension
}
\end{document}

